\chapter{Zaključak i budući rad}
		
		\textbf{\textit{dio 2. revizije}}\\
		
		 \textit{U ovom poglavlju potrebno je napisati osvrt na vrijeme izrade projektnog zadatka, koji su tehnički izazovi prepoznati, jesu li riješeni ili kako bi mogli biti riješeni, koja su znanja stečena pri izradi projekta, koja bi znanja bila posebno potrebna za brže i kvalitetnije ostvarenje projekta i koje bi bile perspektive za nastavak rada u projektnoj grupi.}
		
		 \textit{Potrebno je točno popisati funkcionalnosti koje nisu implementirane u ostvarenoj aplikaciji.}
		

  Projektni zadatak bio je usmjeren na razvoj web aplikacije za praćenje i upravljanje procesima medicinske rehabilitacije, obuhvaćajući vremenski period od 23. listopada 2023. do 19. siječnja 2024. godine. Redovito smo održavali tjedne sastanke i konzultacije tijekom cijelog trajanja projekta kako bismo pratili napredak i usklađivali aktivnosti. Na početku svakog sastanka članovi tima iznosili su svoja postignuća iz prethodnog tjedna, dok smo na kraju sastanka planirali zadatke za sljedeći tjedan.

Prvu fazu projekta obilježila je podjela tima na \textit{frontend} i \textit{backend} skupine. U prvom tjednu posvetili smo se učenju Spring Boota i Reacta, ovisno o odabranoj skupini. Studenti koji nisu bili upoznati s Gitom ili GitHubom prošli su odgovarajuće vježbe. Nakon stjecanja temeljnog znanja, fokusirali smo se na razvoj ideja i mogućnosti aplikacije te definiranje funkcionalnih zahtjeva.

Suradnja unutar tima bila je ključna u oblikovanju funkcionalnosti aplikacije. Veći dio tima posvetili smo i dokumentaciji kako bismo je temeljito pripremili. Prvi izazov dogodio se sredinom studenog kada smo naišli na problem na GitHubu vezan uz nepravilno usklađivanje sekundarnih grana s glavnom granom. Sekundarne grane već su imale dodatne funkcionalnosti, ali su bile u zaostatku u odnosu na glavnu granu. Nakon uspješnog rješavanja problema, unaprijedili smo svoje vještine u radu s Gitom.

\textit{Frontend} tim koristio je alat Figma za oblikovanje izgleda aplikacije, dok se \textit{backend} tim usredotočio na implementaciju registracije, prijave i povezivanje s bazom podataka. \textit{Frontend} tim uspješno je izvršio implementaciju svog dijela koristeći React, pridonoseći ciljevima za prvu predaju projekta. Zadaci vezani uz dokumentaciju bili su raspoređeni prema kraju prve faze, gdje smo izradili dijagrame poput obrazaca uporabe, sekvencijskih dijagrama, modela baze podataka i dijagrama razreda. Iako smo postigli cilj završetka prije predviđenog roka, posljednji tjedan posvetili smo ispravcima i unapređenjima.

Za početak druge faze projekta održan je redoviti sastanak na kojem smo raspodijelili zadatke i postavili okvirni cilj završetka do 10. siječnja, iako je predviđeni rok za zadnju predaju bio dva tjedna kasnije. Odlučili smo odmah krenuti s radom, a frontend tim koristeći alat Figma oblikovao je dizajn preostalog dijela aplikacije.

Nastavili smo s tjednim sastancima za praćenje napretka. \textit{Backend} i \textit{frontend} timovi održavali su dodatne sastanke nekoliko puta tjedno kako bi riješili moguće probleme i međusobno si pomogli u implementaciji. \textit{Backend} tim predvodio je Ian koji je, s obzirom na svoje iskustvo sa Spring Bootom, dijelio zadatke unutar tima. Nakon implementacije, svaki član bi poslao \textit{pull request}, a Ian bi pregledao i dao povratne informacije o potrebnim ispravkama.

Tijekom izrade aplikacije stalno smo dobivali nove ideje, a svaki put kad smo mislili da se približavamo kraju odlučili smo implementirati dodatne mogućnosti. Implementacija ovih dodatnih značajki zahtijevala je značajan dodatnan rad i na \textit{backendu} i na \textit{frontendu}. Frontend tim bio je izložen dodatnom zadatku osmišljavanja novog dizajna kako bi se integrirale sve nove ideje. Ovaj dinamičan proces kreativnog razvoja doprinio je bogatstvu funkcionalnosti i poboljšanju korisničkog iskustva u konačnom proizvodu. Uključili smo opciju tamnog načina rada za bolju preglednost i implementirali virtualnog asistenta koji pomaže korisnicima u navigaciji aplikacijom i pruža informacije o terapijama. Sve funkcionalnosti definirane na početku uspješno smo implementirali. Kako smo se približavali završetku pojedini članovi \textit{backend} tima vratili su se dokumentaciji i dovršili dijagrame stanja, aktivnosti, komponenata te razmještaja. 

Projekt smo uspješno završili usklađujući naše ideje, rješavajući izazove uz pomoć dobre komunikacije. Svjesno smo se suočavali s izazovima, pratili napredak i prilagođavali se novim idejama tijekom cijelog procesa razvoja. Uloženi trud i međusobna podrška doveli su nas do izvrsne aplikacije koja odražava naše zajedničke napore i posvećenost projektu. 
