%definira klasu dokumenta 
\documentclass[12pt]{report} 

%prostor izmedu naredbi \documentclass i \begin{document} se zove uvod. U njemu se nalaze naredbe koje se odnose na cijeli dokument

%osnovni LaTex ne može riješiti sve probleme, pa se koriste različiti paketi koji olakšavaju izradu željenog dokumenta
\usepackage[croatian]{babel} 
\usepackage{amssymb}
\usepackage{amsmath}
\usepackage{txfonts}
\usepackage{mathdots}
\usepackage{titlesec}
\usepackage{array}
\usepackage{lastpage}
\usepackage{etoolbox}
\usepackage{tabularray}
\usepackage{color, colortbl}
\usepackage{adjustbox}
\usepackage{geometry}
\usepackage[classicReIm]{kpfonts}
\usepackage{hyperref}
\usepackage{fancyhdr}
\usepackage{placeins}

\usepackage{float}
\usepackage{setspace}
\restylefloat{table}


\patchcmd{\chapter}{\thispagestyle{plain}}{\thispagestyle{fancy}}{}{} %redefiniranje stila stranice u paketu fancyhdr

%oblik naslova poglavlja
\titleformat{\chapter}{\normalfont\huge\bfseries}{\thechapter.}{20pt}{\Huge}
\titlespacing{\chapter}{0pt}{0pt}{40pt}


\linespread{1.3} %razmak između redaka

\geometry{a4paper, left=1in, top=1in,}  %oblik stranice

\hypersetup{ colorlinks, citecolor=black, filecolor=black, linkcolor=black,	urlcolor=black }   %izgled poveznice


%prored smanjen između redaka u nabrajanjima i popisima
\newenvironment{packed_enum}{
	\begin{enumerate}
		\setlength{\itemsep}{0pt}
		\setlength{\parskip}{0pt}
		\setlength{\parsep}{0pt}
	}{\end{enumerate}}

\newenvironment{packed_item}{
	\begin{itemize}
		\setlength{\itemsep}{0pt}
		\setlength{\parskip}{0pt}
		\setlength{\parsep}{0pt}
	}{\end{itemize}}




%boja za privatni i udaljeni kljuc u tablicama
\definecolor{LightBlue}{rgb}{0.9,0.9,1}
\definecolor{LightGreen}{rgb}{0.9,1,0.9}

%Promjena teksta za dugačke tablice
\DefTblrTemplate{contfoot-text}{normal}{Nastavljeno na idućoj stranici}
\SetTblrTemplate{contfoot-text}{normal}
\DefTblrTemplate{conthead-text}{normal}{(Nastavljeno)}
\SetTblrTemplate{conthead-text}{normal}
\DefTblrTemplate{middlehead,lasthead}{normal}{Nastavljeno od prethodne stranice}
\SetTblrTemplate{middlehead,lasthead}{normal}

%podesavanje zaglavlja i podnožja

\pagestyle{fancy}
\lhead{Programsko inženjerstvo}
\rhead{Medicinska rehabilitacija}
\lfoot{MedBay}
\cfoot{stranica \thepage/\pageref{LastPage}}
\rfoot{\today}
\renewcommand{\headrulewidth}{0.2pt}
\renewcommand{\footrulewidth}{0.2pt}


\begin{document} 
\chapter{Dnevnik promjena dokumentacije}
		
		\textbf{\textit{Kontinuirano osvježavanje}}\\
				
		
		\begin{longtblr}[
				label=none
			]{
				width = \textwidth, 
				colspec={|X[2]|X[13]|X[3]|X[3]|}, 
				rowhead = 1
			}
			\hline
			\textbf{Rev.}	& \textbf{Opis promjene/dodatka} & \textbf{Autori} & \textbf{Datum}\\[3pt] \hline
			0.1 & Napravljena inicijalna skica toka i obrazaca upotrebe. & svi & 23.10.2023. 		\\[3pt] \hline 
			0.2	& Dodani funkcionalni zahtjevi.\newline Dodani obrasci uporabe. & Tea, Nikola, Ivan & 05.11.2023. 	\\[3pt] \hline
            0.2.1 & Revizija funkcionalnih zahtjeva i obrazaca uporabe. & Karlo & 07.11.2023. 	\\[3pt] \hline
			0.4 & Dodan \textit{Use Case} dijagram i jedan sekvencijski dijagram, funkcionalni i nefunkcionalni zahtjevi i dodatak A \newline Dodatna revizija obrazaca uporabe. & Niko, Ivan & 09.11.2023. \\[3pt] \hline 
			0.5 & Dodan opis baze podataka. & Tea & 12.11.2023. \\[3pt] \hline 
            0.6 & Dodan opis arhitekture. & Tea, Karlo & 14.11.2023. \\[3pt] \hline 
			0.8 & Napisan \textit{Opis projektnog zadatka}, \textit{Dodatak}, \textit{Nefunkcionalni zahtjevi} i \textit{Dnevnik promjena} & Karlo & 14.11.2023. \\[3pt] \hline 
			0.9 & Dodani kompletno UC i sekvenijski dijagrami & Niko, Ivan & 16.11.2023. \\[3pt] \hline 
			0.10 & Preveden uvod & * & 08.09.2013. \\[3pt] \hline 
			0.11 & Sekvencijski dijagrami & * & 09.09.2013. \\[3pt] \hline 
			0.12.1 & Započeo dijagrame razreda & * & 10.09.2013. \\[3pt] \hline 
			0.12.2 & Nastavak dijagrama razreda & * & 11.09.2013. \\[3pt] \hline 
			\textbf{1.0} & Verzija samo s bitnim dijelovima za 1. ciklus & * & 11.09.2013. \\[3pt] \hline 
			1.1 & Uređivanje teksta -- funkcionalni i nefunkcionalni zahtjevi & * \newline * & 14.09.2013. \\[3pt] \hline 
			1.2 & Manje izmjene:Timer - Brojilo vremena & * & 15.09.2013. \\[3pt] \hline 
			1.3 & Popravljeni dijagrami obrazaca uporabe & * & 15.09.2013. \\[3pt] \hline 
			1.5 & Generalna revizija strukture dokumenta & * & 19.09.2013. \\[3pt] \hline 
			1.5.1 & Manja revizija (dijagram razmještaja) & * & 20.09.2013. \\[3pt] \hline 
			\textbf{2.0} & Konačni tekst predloška dokumentacije  & * & 28.09.2013. \\[3pt] \hline 
			&  &  & \\[3pt] \hline	
		\end{longtblr}
	
	
		\textit{Moraju postojati glavne revizije dokumenata 1.0 i 2.0 na kraju prvog i drugog ciklusa. Između tih revizija mogu postojati manje revizije već prema tome kako se dokument bude nadopunjavao. Očekuje se da nakon svake značajnije promjene (dodatka, izmjene, uklanjanja dijelova teksta i popratnih grafičkih sadržaja) dokumenta se to zabilježi kao revizija. Npr., revizije unutar prvog ciklusa će imati oznake 0.1, 0.2, …, 0.9, 0.10, 0.11.. sve do konačne revizije prvog ciklusa 1.0. U drugom ciklusu se nastavlja s revizijama 1.1, 1.2, itd.}

\end{document}

