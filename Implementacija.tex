\chapter{Implementacija i korisničko sučelje}
		
		
		\section{Korištene tehnologije i alati}
		
			\textbf{\textit{dio 2. revizije}}
			
			 \textit{Detaljno navesti sve tehnologije i alate koji su primijenjeni pri izradi dokumentacije i aplikacije. Ukratko ih opisati, te navesti njihovo značenje i mjesto primjene. Za svaki navedeni alat i tehnologiju je potrebno \textbf{navesti internet poveznicu} gdje se mogu preuzeti ili više saznati o njima}.
			
\textbf{	KORIŠTENE TEHNOLOGIJE
} 

Prilikom izrade web aplikacija za upravljanje rehabilitacijom koristili smo se raznim tehnologijama kako bismo pružili korisnicima učinkovit i interaktivan doživljaj. Backend aplikacije razvili smo koristeći \href{https://www.spring.io/projects/spring-boot}{Spring Boot}, \href{https://www.oracle.com/java}{Java} bazirani framework, koji omogućava brz i jednostavan razvoj server-side logike. \href{https://www.oracle.com/java}{Java}, kao programski jezik, doprinosi robustnosti i skalabilnosti našeg backend sustava.

Za dinamičke i interaktivne elemente na korisničkom sučelju, koristili smo \href{https://www.developer.mozilla.org/en-US/docs/Web/JavaScript}{JavaScript}, dok je \href{https://www.reactjs.org}{React}, popularni \href{https://www.developer.mozilla.org/en-US/docs/Web/JavaScript}{JavaScript} framework, omogućio izgradnju efikasnog korisničkog sučelja. \href{https://www.reactjs.org}{React}, koristeći koncept komponenti, pojednostavljuje organizaciju i održavanje koda, pridonoseći poboljšanju korisničkog iskustva i olakšavajući upravljanje stanjem naše aplikacije 

Podaci o pacijentima, terapijama i terminima pohranjeni su u \href{https://www.postgresql.org}{PostgreSQL} bazi podataka koja pruža pouzdanu podršku. . \href{https://www.postgresql.org}{PostgreSQL}, kao snažan objektno-relacijski sustav upravljanja bazama podataka, omogućava nam efikasno upravljanje informacijama uz podršku za kompleksne upite i transakcije. Osim toga, \href{https://www.developer.mozilla.org/en-US/docs/Web/HTML}{HTML} i \href{https://www.developer.mozilla.org/en-US/docs/Web/CSS}{CSS} koriste se za strukturiranje sadržaja web stranice i stilizaciju, stvarajući tako funkcionalno i atraktivno korisničko sučelje.

Dokumentacija projekta oblikovana je pomoću \href{https://www.overleaf.com/learn/latex/Learn_LaTeX_in_30_minutes}{LaTeX} sustava za pripremu dokumenata, pružajući precizno formatiranje i organizaciju. Za organizaciju sastanaka i suradnju koristili smo \href{https://www.markdownguide.org}{Markdown} format, pružajući jednostavan i čitljiv način za pisanje i dijeljenje informacija.

U procesu zajedničkog rada, pisanja koda i praćenja promjena koristili smo \href{https://git-scm.com/}{Git}. Omogućavao je stvaranje, pregledavanje i spajanje promjena koda, čime se postizavao učinkovit timski rad. Sve ove tehnologije integrirane su u našem projektu kako bismo osigurali da interakcija između bolesnika, djelatnika zdravstvene ustanove i administratora bude glatka, učinkovita i prilagođena specifičnim ulogama i funkcionalnim zahtjevima svakog dionika.



\textbf{KORIŠTENI ALATI
}

U procesu razvoja naše aplikacije koristili smo raznovrsne alate kako bismo unaprijedili različite aspekte projekta. Alat \href{https://www.heroku.com}{Heroku}, poznat po svojoj cloud platformi, omogućio nam je jednostavno deployanje, skaliranje i učinkovito upravljanje web aplikacijama. Korištenjem \href{https://www.heroku.com}{Heroku}-a, značajno smo pojednostavili proces implementacije i održavanja naše aplikacije.

Za potrebe pisanja, testiranja i debugiranja koda koristili smo \href{https://www.visualstudio.com}{Visual Studio Code} (\href{https://www.visualstudio.com}{VSCode}), integrirano razvojno okruženje (IDE). \href{https://www.visualstudio.com}{VSCode} je pružio alate koji su doprinijeli efikasnosti našeg tima tijekom razvoja aplikacije. \href{https://www.jetbrains.com/idea}{IntelliJ IDEA}, kao razvojno okruženje specifično za Java programski jezik, optimizirao je rad na backend dijelu naše aplikacije. 

\href{https://github.com}{GitHub}, platforma za upravljanje verzijama koda i suradnju timova, poslužila nam je za praćenje promjena, upravljanje zadacima te implementaciju pull requestova, unapređujući suradnju tima. Za brzu i jednostavnu komunikaciju te video razgovore koristili smo \href{https://discord.com}{Discord}, pružajući središnje mjesto za razmjenu informacija među članovima tima.


\href{https://www.notion.so}{Notion}, kao alat za organizaciju zadataka, vođenje bilješki sastanaka i općenito upravljanje projektom, korišten je kako bi pridonio boljoj organizaciji i praćenju napretka. \href{https://www.overleaf.com}{Overleaf}, online platforma za suradničko pisanje dokumenata u LaTeX formatu, olakšala je stvaranje i uređivanje dokumentacije našeg projekta.
\href{https://www.visual-paradigm.com}{Visual Paradigm} pruža alate za izradu različitih dijagrama, a mi smo ga koristili za izradu dijagrama obrasca uporabe, sekvencijskih dijagrama, dijagrama stanja, aktivnosti i komponenata potrebnih za dokumentaciju. 

\href{https://www.figma.com}{Figma}, alat za dizajn, poslužio nam je za suradnju u izradi vizualnih planova naše web aplikacije. Kroz Figma-u definirali smo izgled frontend dijela naše aplikacije. \href{https://www.adobe.com/products/premiere.html}{Adobe Premiere Pro} i \href{https://www.adobe.com/products/audition.html}{Audition} koristili smo za video i audio produkciju, stvarajući marketinške materijale i prezentacije vezane uz našu aplikaciju.

Naša interakcija s \href{https://openai.com/gpt}{ChatGPT}-om bila je višestruko korisna, pružajući podršku u različitim aspektima, uključujući punjenje baze podataka te pružanje informacija i savjeta u tijeku razvoja projekta. 


 		Napomena: Prilikom klika u tekstu na nazive korištenih alata i tehnologija otvara se internet poveznica na kojoj se može saznati više informacija o njima.
			\eject 
		
	
		\section{Ispitivanje programskog rješenja}
			
			\textbf{\textit{dio 2. revizije}}\\
			
			 \textit{U ovom poglavlju je potrebno opisati provedbu ispitivanja implementiranih funkcionalnosti na razini komponenti i na razini cijelog sustava s prikazom odabranih ispitnih slučajeva. Studenti trebaju ispitati temeljnu funkcionalnost i rubne uvjete.}
	
			
			\subsection{Ispitivanje komponenti}
			\textit{Potrebno je provesti ispitivanje jedinica (engl. unit testing) nad razredima koji implementiraju temeljne funkcionalnosti. Razraditi \textbf{minimalno 6 ispitnih slučajeva} u kojima će se ispitati redovni slučajevi, rubni uvjeti te izazivanje pogreške (engl. exception throwing). Poželjno je stvoriti i ispitni slučaj koji koristi funkcionalnosti koje nisu implementirane. Potrebno je priložiti izvorni kôd svih ispitnih slučajeva te prikaz rezultata izvođenja ispita u razvojnom okruženju (prolaz/pad ispita). }
			
			
			
			\subsection{Ispitivanje sustava}
			
			 \textit{Potrebno je provesti i opisati ispitivanje sustava koristeći radni okvir Selenium\footnote{\url{https://www.seleniumhq.org/}}. Razraditi \textbf{minimalno 4 ispitna slučaja} u kojima će se ispitati redovni slučajevi, rubni uvjeti te poziv funkcionalnosti koja nije implementirana/izaziva pogrešku kako bi se vidjelo na koji način sustav reagira kada nešto nije u potpunosti ostvareno. Ispitni slučaj se treba sastojati od ulaza (npr. korisničko ime i lozinka), očekivanog izlaza ili rezultata, koraka ispitivanja i dobivenog izlaza ili rezultata.\\ }
			 
			 \textit{Izradu ispitnih slučajeva pomoću radnog okvira Selenium moguće je provesti pomoću jednog od sljedeća dva alata:}
			 \begin{itemize}
			 	\item \textit{dodatak za preglednik \textbf{Selenium IDE} - snimanje korisnikovih akcija radi automatskog ponavljanja ispita	}
			 	\item \textit{\textbf{Selenium WebDriver} - podrška za pisanje ispita u jezicima Java, C\#, PHP koristeći posebno programsko sučelje.}
			 \end{itemize}
		 	\textit{Detalji o korištenju alata Selenium bit će prikazani na posebnom predavanju tijekom semestra.}
			
			\eject 
		
		
		\section{Dijagram razmještaja}
			
			\textbf{\textit{dio 2. revizije}}
			
			 \textit{Potrebno je umetnuti \textbf{specifikacijski} dijagram razmještaja i opisati ga. Moguće je umjesto specifikacijskog dijagrama razmještaja umetnuti dijagram razmještaja instanci, pod uvjetom da taj dijagram bolje opisuje neki važniji dio sustava.}
			
			\eject 
		
		\section{Upute za puštanje u pogon}
		
			\textbf{\textit{dio 2. revizije}}\\
		
			 \textit{U ovom poglavlju potrebno je dati upute za puštanje u pogon (engl. deployment) ostvarene aplikacije. Na primjer, za web aplikacije, opisati postupak kojim se od izvornog kôda dolazi do potpuno postavljene baze podataka i poslužitelja koji odgovara na upite korisnika. Za mobilnu aplikaciju, postupak kojim se aplikacija izgradi, te postavi na neku od trgovina. Za stolnu (engl. desktop) aplikaciju, postupak kojim se aplikacija instalira na računalo. Ukoliko mobilne i stolne aplikacije komuniciraju s poslužiteljem i/ili bazom podataka, opisati i postupak njihovog postavljanja. Pri izradi uputa preporučuje se \textbf{naglasiti korake instalacije uporabom natuknica} te koristiti što je više moguće \textbf{slike ekrana} (engl. screenshots) kako bi upute bile jasne i jednostavne za slijediti.}
			
			
			 \textit{Dovršenu aplikaciju potrebno je pokrenuti na javno dostupnom poslužitelju. Studentima se preporuča korištenje neke od sljedećih besplatnih usluga: \href{https://aws.amazon.com/}{Amazon AWS}, \href{https://azure.microsoft.com/en-us/}{Microsoft Azure} ili \href{https://www.heroku.com/}{Heroku}. Mobilne aplikacije trebaju biti objavljene na F-Droid, Google Play ili Amazon App trgovini.}
			
			
			\eject 